% ------------- PREAMBLE ------------- %
% Currently using IEEEtran 1.8
\documentclass[10pt,conference,a4paper]{IEEEtran}
\usepackage{amsmath}
\usepackage{ucs}
\usepackage[utf8x]{inputenc}
\usepackage{times}
\usepackage{graphicx}
\setcounter{MaxMatrixCols}{30}
\usepackage{amsfonts}
\usepackage{amssymb}
\usepackage{url}

% Style
\newcommand{\CLASSINPUTinnersidemargin}{18mm}
\newcommand{\CLASSINPUToutersidemargin}{12mm}
\newcommand{\CLASSINPUTtoptextmargin}{20mm}
\newcommand{\CLASSINPUTbottomtextmargin}{25mm}
\DeclareGraphicsExtensions{.png,.eps,.ps,.pdf,.jpg}
% Hack para corregir las cabeceras de tablas de IEEEtran. Con esto figuras y tablas tienen el mismo formato.
\makeatletter \def\@IEEEtablestring{} \makeatother

% ------------- DOCUMENT ------------- %
\title{CARACTERIZACIÓN DE MEDIOS DE TRANSMISIÓN}
\author{
    \IEEEauthorblockN{Pablo Arrieta Nata, Daniel Iñigo Baños.}
    \IEEEauthorblockA{E-mail Primer Autor, uo194823@uniovi.es.}
    \IEEEauthorblockA{Grado en Tecnologias y Servicios de Telecomunicacion. Universidad de Oviedo.}
    \iEEEauthorblockA{Sistemas de Telecomunicación. Curso 2013-14.}
}
\begin{document}
\maketitle

\begin{abstract}
    Los medios de transmisión guiados y no guiados, como son el cable coaxial y el canal WiFi, presentan ventajas unos frente a otros en diferentes aspectos. Por esto, dependiendo del tipo de transmisión que se quiera realizar se emplea uno u otro. A continuación se estudian independientemente y se realiza una breve comparación a partir de su análisis en frecuencia.
\end{abstract}

% ------------------------------------------- %
% ------------- PRIMER MONTAJE  ------------- %
% ------------------------------------------- %
\section{Primer Montaje}
Empleando un analizador de redes se mide la respuesta en frecuencia de un cable coaxial mediante el parámetro &s_{2,1}& en el ancho de banda de 1[MHz] a 2[GHz]. Se transmiten 0[dBm], que es 1[mW]. El &s_{2,1}& representa la función de transferencia del cable en módulo-fase.

\subsection{Figuras y Tablas}

%\begin{figure}[htb]
    %\centering
    %\includegraphics[width=\columnwidth]{folder/widthfile.eps}
    %\caption{This is the caption text}
    %\label{fig:graph}
%\end{figfigure}
% -------------------------------------------- %
% ------------- SEGUNDO MONTAJE  ------------- %
% -------------------------------------------- %
\section{Segundo Montaje}

\subsection{Figuras y Tablas}

El tamaño para los títulos de las tablas, figuras y notas al pie de página es
de 8 puntos. Todas las figuras y tablas aparecerán centradas en la columna
(las figuras y tablas de gran tamaño podrán extenderse sobre ambas columnas);
evite ubicarlas en medio de las columnas. La descripción de las figuras y las
tablas deberá ubicarse debajo de las mismas. Use la abreviatura Fig. x para
referirse a una figura o gráfico y Tabla x para referirse a una tabla. Los
pies de las figuras y de las tablas deben seguir el formato mostrado bajo la
Fig. 1.

%Ejemplo de Tabla:
%\begin{table}
%\renewcommand{\arraystretch}{1.3}
%\caption{An Example of a Table}
%\label{table_example}
%\begin{center}
%\begin{tabular}{|c||c|}
%\hline
%One & Two\\
%\hline
%Three & Four\\
%\hline
%\end{tabular}
%\end{center}
%\end{table}

%\begin{figure}[htb]
    %\centering
    %\includegraphics[width=\columnwidth]{folder/widthfile.eps}
    %\caption{This is the caption text}
    %\label{fig:graph}
%\end{figfigure}

\subsection{Ecuaciones}

Las ecuaciones matemáticas deben estar situadas en líneas distintas y cada
ecuación matemática debe ser numerada:
\begin{equation}
    \underset{t\rightarrow\infty}{\lim}x(t)=\frac{s[f_{\omega}(t)]}{\cos
    ^{-1}\theta}
\end{equation}

\subsection{Numeración de páginas}

No aplique ninguna numeración de página. La numeración se añadirá en el
proceso final de confección de las actas en CD-ROM. Por favor deje la
numeración tal como está en el documento modelo.

\subsection{Referencias}

Las referencias serán numeradas en orden de aparición [1]. El formato de
referencias será el estándar del IEEE. Se muestra algún ejemplo en el apartado correspondiente.

\subsection{Uno o más autores}

En caso de tener uno, dos o más de tres autores, adapte la zona
correspondiente a autores y afiliación de manera oportuna. Intente no variar
de manera notable el aspecto y tamaño de la zona.

\section{Conclusiones}

El seguimiento de las normas indicadas permitirá que su trabajo resulte
visualmente atractivo y que de lugar a impresiones de calidad. Esta misma
plantilla se puede encontrar en los formatos OpenDocument (ISO/IEC 26300:2006)
y Microsoft Word\textsuperscript{\textregistered} en la dirección web oficial del Simposium.

%Bibliografía

\begin{thebibliography}{9}                                                                                                %
    \bibitem {ref libro}C. Jones and K. Jones, \emph{How to publish a paper in 30
	seconds,} 233rd ed., It is a Joke, Republic of Chiquitistan, 2009.

    \bibitem {ref articulo}K. Jones and L.Grijander, \textquotedblleft How to use
	the copy and paste tool\textquotedblright\ Trans. on Phys. Rev., vol. 234, no.
	7, pp. 635-646, Dec. 1935.
\end{thebibliography}


\end{document}
