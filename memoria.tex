%Ejemplo de Paper para la URSI 2010
% respecto a plantillas anteriores cambio de latin1 a utf8
% para que salgan las cosas en castellano...
% para que salgan las cosas en castellano...
% para que salgan las cosas en castellano...
% para que salgan las cosas en castellano...
% Fuente Times...
% figuras en formato .png, .ps, pdf ó eps
% Fuente Times...
% figuras en formato .png, .ps, pdf ó eps
% Fuente Times...
% figuras en formato .png, .ps, pdf ó eps
% Fuente Times...
% figuras en formato .png, .ps, pdf ó eps
%\usepackage[spanish]{babel}

\documentclass[10pt,conference,a4paper]{IEEEtran}%
\usepackage{amsmath}
\usepackage{ucs}
\usepackage[utf8x]{inputenc}
\usepackage{times}
\usepackage{graphicx}%
\setcounter{MaxMatrixCols}{30}%
\usepackage{amsfonts}%
\usepackage{amssymb}
\usepackage{url}
%TCIDATA{OutputFilter=latex2.dll}
%TCIDATA{Version=5.00.0.2606}
%TCIDATA{LastRevised=Friday, February 13, 2009 11:03:49}
%TCIDATA{<META NAME="GraphicsSave" CONTENT="32">}
%TCIDATA{<META NAME="SaveForMode" CONTENT="1">}
%TCIDATA{BibliographyScheme=Manual}
\newcommand{\CLASSINPUTinnersidemargin}{18mm}
\newcommand{\CLASSINPUToutersidemargin}{12mm}
\newcommand{\CLASSINPUTtoptextmargin}{20mm}
\newcommand{\CLASSINPUTbottomtextmargin}{25mm}
\DeclareGraphicsExtensions{.png,.eps,.ps,.pdf}
\begin{document}

\title{COLOQUE AQUÍ EL TÍTULO}
\author{\authorblockN{Nombre del Primer Autor, Nombre del Segundo
Autor.} \authorblockA{E-mail Primer Autor, E-mail Segundo Autor.}
\authorblockA{Grado en Tecnologias y Servicios de Telecomunicacion. Universidad de Oviedo.}
\authorblockA{Sistemas de Telecomunicaci�n. Curso 2013-14.} }
\maketitle

\begin{abstract}
Este documento presenta la plantilla para el XXV Simposium Nacional de la
Unión Científica Internacional de Radio (URSI). La Universidad del País Vasco/Euskal Herriko
Unibertsitatea (UPV/EHU) organiza la presente edición de la URSI. El abstract
o resumen de la comunicación debe estar en \underline{INGLÉS} y debe contener
entre 100 y 150 palabras. El tipo de letra para el abstract es Times New Roman
en negrita con tamaño 9 puntos, tal y como aparece en este documento. Esta
plantilla ha sido generada tomando como base el estándar de documentos IEEE.
Por favor, utilice la misma como documento base para el trabajo que desee
enviar a URSI 2010. Cualquier cuestión sobre el envío de los artículos debe
dirigirse a la secretaría técnica del congreso (\url{ponencias@ursi2010.org}).
Para más información sobre la organización y desarrollo del congreso se puede
consultar la dirección: \url{http://www.ursi2010.org}.

\end{abstract}

\section{Introducción}

Esta plantilla puede encontrarse en el sitio web dedicado al simposium. Este
documento es un ejemplo del formato de presentación deseado, y contiene
información concerniente al diseño general del documento, tipo de letra, y
tamaños de tipografía apropiados. El cuerpo del artículo se escribirá en
castellano. El trabajo tendrá una extensión no superior a 4 páginas. Este
fichero debería servir como punto de partida para los artículos enviados a
URSI 2010 producidos bajo \LaTeX\ usando IEEEtran.cls versión 1.6 y posteriores.

\section{Formato}

Utilice tipografía Times New Roman. El tamaño para el cuerpo del texto es de
10 puntos y para el Título del artículo 24 puntos. Utilice formato A4 (21 x
29,7 cm), ajuste los márgenes superior e inferior a 2 y 2,5 cm
respectivamente, el margen izquierdo a 1,8 cm y el derecho a 1,2 cm. El
artículo deberá ir a dos columnas con un espaciado entre columnas de 0,42 cm.
Justifique las columnas tanto a izquierda como a derecha. Los párrafos deberán
ser escritos a simple espacio.

\subsection{Figuras y Tablas}

El tamaño para los títulos de las tablas, figuras y notas al pie de página es
de 8 puntos. Todas las figuras y tablas aparecerán centradas en la columna
(las figuras y tablas de gran tamaño podrán extenderse sobre ambas columnas);
evite ubicarlas en medio de las columnas. La descripción de las figuras y las
tablas deberá ubicarse debajo de las mismas. Use la abreviatura Fig. x para
referirse a una figura o gráfico y Tabla x para referirse a una tabla. Los
pies de las figuras y de las tablas deben seguir el formato mostrado bajo la
Fig. 1.

%Ejemplo de Tabla:
%\begin{table}
%\renewcommand{\arraystretch}{1.3}
%\caption{An Example of a Table}
%\label{table_example}
%\begin{center}
%\begin{tabular}{|c||c|}
%\hline
%One & Two\\
%\hline
%Three & Four\\
%\hline
%\end{tabular}
%\end{center}
%\end{table}

\begin{figure}[pth]
\centering\includegraphics [width=8.7 cm]
{Identidad_Grafica_URSI.png}\caption{Identidad Gráfica de la URSI 2010}%
\end{figure}

\subsection{Ecuaciones}

Las ecuaciones matemáticas deben estar situadas en líneas distintas y cada
ecuación matemática debe ser numerada:
\begin{equation}
\underset{t\rightarrow\infty}{\lim}x(t)=\frac{s[f_{\omega}(t)]}{\cos
^{-1}\theta}%
\end{equation}

\subsection{Numeración de páginas}

No aplique ninguna numeración de página. La numeración se añadirá en el
proceso final de confección de las actas en CD-ROM. Por favor deje la
numeración tal como está en el documento modelo.

\subsection{Referencias}

Las referencias serán numeradas en orden de aparición [1]. El formato de
referencias será el estándar del IEEE. Se muestra algún ejemplo en el apartado correspondiente.

\subsection{Uno o más autores}

En caso de tener uno, dos o más de tres autores, adapte la zona
correspondiente a autores y afiliación de manera oportuna. Intente no variar
de manera notable el aspecto y tamaño de la zona.

\section{Conclusiones}

El seguimiento de las normas indicadas permitirá que su trabajo resulte
visualmente atractivo y que de lugar a impresiones de calidad. Esta misma
plantilla se puede encontrar en los formatos OpenDocument (ISO/IEC 26300:2006)
y Microsoft Word\textsuperscript{\textregistered} en la dirección web oficial del Simposium.

%Bibliografía

\begin{thebibliography}{9}                                                                                                %
\bibitem {ref libro}C. Jones and K. Jones, \emph{How to publish a paper in 30
seconds,} 233rd ed., It is a Joke, Republic of Chiquitistan, 2009.

\bibitem {ref articulo}K. Jones and L.Grijander, \textquotedblleft How to use
the copy and paste tool\textquotedblright\ Trans. on Phys. Rev., vol. 234, no.
7, pp. 635-646, Dec. 1935.
\end{thebibliography}


\end{document}
